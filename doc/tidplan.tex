\documentclass{TDP003mall}



\newcommand{\version}{Version 1,1}
\author{Adam Ivarsson, \url{adaiv505@student.liu.se}\\
  Lukas Michanek, \url{lukmi182@student.liu.se}}
\title{Tidplan}
\date{2015-09-08}
\rhead{Adam Ivarsson\\
Lukas michanek\\}



\begin{document}
\projectpage
\section{Planering}
Alla uppsatta timantal är per person. Individuella deadlines och uppgifter, såsom dagboken och muntorna, är ej med
i denna planering.

\section{Arbetsätt}

Under själva arbetet kommer programmeringen ske till stor del med varandra, detta innebär inte nödvändigtvis att vi
samtidigt skriver på samma kod, utan att ena exempelvis arbetar på presentationslagret medans den andra skriver på
datalagret. En person kommer aldrig skriva en hel “modul” på egen hand, utan vi kommer se till att turas om och vara
noggranna med att redovisa vårat arbete för varandra så vi i slutändan båda har bra koll på hur systemet i storhet
fungerar och hur det är uppbyggt.

\section{Veckoplanering}

\subsection{Vecka 36}
\begin{itemize}
\item Planeringsdokument, deadline kl. 23:59 den 03/09.
\end{itemize}

\subsection{Vecka 37}
\begin{itemize}
\item Sätta generella mål, hur högt vill vi nå, samt vad vill vi spendera mer tid på, osv. 1h
\item Brainstorming-pass, med sketching av LoFi-prototyper samt en basplan över hur datalagret ska byggas upp. 2h
\item Färdigställande av LoFi-prototyp samt skrivande av manual. 4h
\item LoFi-prototyp och manual, deadline kl. 23:59 den 10/09.
\end{itemize} 

\subsection{Vecka 38}
\begin{itemize}
\item Noggrannare plan för datalagret och dess uppbyggnad. 1h
\item Konfigurering av revisionshantering, korrekt katalogstruktur, samt installation av ramverk. 1h
\item Påbörja datalagret och front-end. 4h
\item Skrivande av installationsmanualen samt användningsmanualen. 4h
\item Installations och användningsmanual, deadline kl. 23:59 17/09.
\end{itemize}

\subsection{Vecka 39}
\begin{itemize}
\item Fortsättning av datalagret. 4h
\item Bidra till de gemensamma manualerna. 1h
\item Bidrag till den gemensamma installationsmanualen och projektdatabasen, deadline kl. 23:59 den 24/09.
\end{itemize}

\subsection{Vecka 40}
\begin{itemize}
\item Fortsättning och avslutning av datalagret. 5h
\item Redovisa datalagret för assistent den 1/10. 30min
\item Sista korrigeringar i den gemensamma installationsmanualen, deadline kl. 23:59 den 1/10. 30min
\item Eventuell komplettering gällande datalagret den 2/10. 2h
\item Putsa front-end. 1h
\end{itemize}

\subsection{Vecka 41}
\begin{itemize}
\item Putsa front-end. 4h
\item Skrivande av systemdokumentation. 4h
\item Inlämning av systemdokumentation, deadline kl. 23:59 den 8/10.
\end{itemize}

\subsection{Vecka 42}
\begin{itemize}
\item Komplettering av systemdokumentation. 2h
\item Systemtestning och snabb dokumentation av vad som gick fel. 4h
\item Fullt funktionellt projekt den 15/10.
\end{itemize}

\subsection{Vecka 43}
\begin{itemize}
\item Feldokumentation från tidigare noteringar av vad som gick fel. 2h
\item Inlämning av det färdigställda systemet samt testdokumentationen. Alla fel dokumenterade men inte nödvändigtvis korrigerade, deadline kl. 23:59 den 19/10.
\item Reflektionsdokument, deadline kl. 23:59 den 22/10. 4h
\end{itemize}

\subsection{Vecka 44}
\begin{itemize}
\item Slutgiltig inlämning av kod precis innan munta.
\end{itemize}

\section{Prioriteringar}

\subsection{Front-end}
\begin{enumerate}
\item Barebones, ingen CSS, bara för att testa funktionaliteten av datalagret.
\item Portfölj-struktur enligt mall.
\item Enkel layout samt positionering av element.
\item Webbdesign, CSS för att liva upp sidan med färger (t.ex. olika färger för olika språktaggar).
\item Dynamisk index-sida.
\item Finputsning samt graphical fidelity.
\end{enumerate}

\subsection{Back-end}
\begin{enumerate}
\item Grundläggande testfunktionalitet.
\item Första funktionen: Skicka data via Flask
\item Andra funktionen: Hämta data via JSON
\item Första funktionaliteten av mallar
\item Integration med vår barebones front-end
\item Milstolpe: Första funktionella hemsidan
\item Sökfunktionalitet
\item Sorteringsfunktionalitet
\item Milstolpe: Inlämningsbart projekt
\item Inloggning med formulär för att skapa egna inlägg.
\end{enumerate}
\end{document}
